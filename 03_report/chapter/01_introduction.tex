\chapter{Introduction}

The objective of this laboratory exercise is to design and implement a closed-loop controller
for the Ball-in-Tube system, in which the vertical position of a lightweight ping-pong ball is
regulated by an upward airflow generated by a DC fan. The task involves modelling the
nonlinear aerodynamic system, identifying the relevant parameters from experimental data,
deriving a suitable linear approximation for control design, and implementing a robust PID
controller capable of tracking a desired height under realistic constraints.

The system exhibits nonlinear drag behaviour, actuator dynamics with a first-order response,
measurement noise, and physical limitations such as tube boundaries and voltage saturation.
To handle these effects, a combination of nonlinear modelling, state linearization around a
chosen operating point, and classical control design techniques is applied. A trajectory-planning
approach based on differential flatness is additionally used to achieve smooth position changes.

The structure of this report follows the recommended workflow for advanced control
engineering experiments. Chapter~2 provides a concise summary of the theoretical aspects
relevant for modelling, linearization, parameter identification, and trajectory generation.
Chapter~3 describes the laboratory setup and the procedure used to identify key system
constants from measurements. The implemented Simulink model and simulation structure
are presented in Chapter~4. Chapter~5 reports and interprets the resulting simulation and
experimental data, including a comparison between model predictions and real system
behaviour. Finally, Chapter~6 summarizes the main findings, encountered challenges, and
possible improvements for future work.
