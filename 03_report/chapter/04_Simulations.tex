\chapter{Linearization and Parameter Identification}
The nonlinear state-space model is linearized around a selected operating point to enable classical controller design. The resulting linear system matrices and transfer functions are derived. In addition, this chapter explains how the system parameters such as $\alpha_i$, $k_\omega$, $\tau_\omega$, and the voltage offset $u_0$ are obtained through measurement data, steady-state evaluation, or provided laboratory constants.


% % Figures:
% \begin{figure}[!t]
%   \begin{tikzpicture}
%     \begin{axis}[width=15cm, height=6cm,
%       minor tick num   = 2,
%       grid             = both,	% have a grid that is bitchin
%       % GRID OPTIONS
%       minor grid style = {
%         densely dotted,
%         line width = 0.1,
%         gray!30
%       },
%       major grid style={
%         densely dotted,
%         line width = 0.3,
%         gray!30
%       }]
%       \addplot+ [
%       const plot mark right,
%       ] coordinates {
%         (0,0.1)    (0.1,0.15)  (0.2,0.5)   (0.3,0.62)
%         (0.4,0.56) (0.5,0.58)  (0.6,0.65)  (0.7,0.6)
%         (0.8,0.58) (0.9,0.55)  (1,0.52)
%       };
%     \end{axis}
%   \end{tikzpicture}
%   \caption{The authors interest in the topic as years go on.}
% \end{figure}

% % Figures:
% \begin{figure}[!t]
%   \begin{tikzpicture}
%     \begin{axis}[width=15cm, height=6cm,
%       minor tick num   = 2,
%       grid             = both,	% have a grid that is bitchin
%       % GRID OPTIONS
%       minor grid style = {
%         densely dotted,
%         line width = 0.1,
%         gray!30
%       },
%       major grid style={
%         densely dotted,
%         line width = 0.3,
%         gray!30
%       }]
%       \addplot+ [
%       const plot mark right,
%       ] coordinates {
%         (0,0.1)    (0.1,0.15)  (0.2,0.5)   (0.3,0.62)
%         (0.4,0.56) (0.5,0.58)  (0.6,0.65)  (0.7,0.6)
%         (0.8,0.58) (0.9,0.55)  (1,0.52)
%       };
%     \end{axis}
%   \end{tikzpicture}
%   \caption{The authors interest in the topic as years go on.}
% \end{figure}