\chapter{Theoretical Background}

\begin{figure}[h!]
    \centering
    \includegraphics[width=0.40\textwidth]{img/bit_physical_overview.png}
    \caption{Schematic representation of the Ball-in-Tube system}
    \label{fig:bit_physical}
\end{figure}

The Ball-in-Tube system consists of a ping-pong ball inside a vertical tube. An upward air flow, 
generated by a fan, interacts with the ball and creates a levitating equilibrium. The geometry 
of the tube and the ball defines the cross-sectional areas relevant for the airflow model, while 
the fan dynamics introduce a delay between the applied voltage and the resulting air velocity.

Ball motion results from its mass, the gravitational force and the pressure drag acting on its 
surface. Frictional drag is negligible for this type of voluminous body and identical air 
density in both tube and gap is assumed. With these simplifications the drag force depends on 
the relative velocity between airflow and ball and leads to the nonlinear model
\[
    \ddot{z}(t) = (\alpha_1 \omega(t) - \alpha_2 \dot{z}(t))^2 - \alpha_3
\]
where \(\alpha_1\), \(\alpha_2\) and \(\alpha_3\) combine the physical parameters of the 
system into compact coefficients.

Airflow dynamics are shaped by the behaviour of the fan. An electrical model with resistance, 
inductance and a voltage term proportional to the motor speed forms the basis for its 
description, as shown in Fig.~\ref{fig:bit_electrical}. Motor inductance is small in the 
operating range of interest and is therefore neglected. With this simplification the fan 
behaves as a first-order system that relates the input voltage \(u(t)\) to the fan speed 
\(\omega(t)\)
\[
    \tau_\omega \dot{\omega}(t) + \omega(t) = k_\omega (u(t) - u_0)
\]
with gain \(k_\omega\), time constant \(\tau_\omega\) and voltage offset \(u_0\).

\begin{figure}[h!]
    \centering
    \includegraphics[width=0.50\textwidth]{img/bit_electrical_model.png}
    \caption{Electrical model of the fan motor}
    \label{fig:bit_electrical}
\end{figure}

Coupling the aerodynamic model with the fan dynamics yields a nonlinear description of the 
overall system. A convenient set of state variables uses ball height, vertical velocity and 
fan speed, written compactly as \(x = [z, \dot{z}, \omega]^\top\). Nonlinearities in the lift 
term and the indirect influence of the control input make direct controller design on this 
model impractical. Linearisation around a steady operating point provides a local 
approximation that supports classical PID design in the frequency domain.

PID control forms the central regulation strategy in this laboratory. Nonlinear effects, 
saturation limits and the relatively slow fan response influence achievable performance and 
need to be considered during tuning. For smooth height transitions a flatness-based 
feed-forward term can be derived, since the ball position \(z(t)\) acts as a flat output of 
the system. This allows the generation of reference trajectories and matching feed-forward 
inputs with the required degree of smoothness.



% % Take a look at the lab_BiT.pdf file for reference!
% \begin{itemize}
%     \item Physical description of the system: ball, airflow, tube geometry, fan. % Should we add a Picture or Grafic on that like in the Assignment?
%     \item Forces acting on the ball: gravity and pressure drag force. % Should we add a Picture or Grafic on that like in the Assignment?
%     \item Derivation of the nonlinear ball dynamics:
%     \[
%         \ddot{z}(t) = (\alpha_1 \omega(t) - \alpha_2 \dot{z}(t))^2 - \alpha_3.
%     \]
%     \item Relative velocity formulation and influence on drag force.
%     \item Derivation of the fan model and simplification to a PT1 system:
%     \[
%         \tau_\omega \dot{\omega}(t) + \omega(t) = k_\omega (u(t) - u_0).
%     \]
%     \item Combined nonlinear state-space representation with states \(x = [z, \dot{z}, \omega]^\top\).
%     \item Motivation for linearization for controller design.
%     \item Fundamentals of PID control and its challenges in nonlinear systems.
%     \item Differential flatness of the system and motivation for trajectory planning.
% \end{itemize}