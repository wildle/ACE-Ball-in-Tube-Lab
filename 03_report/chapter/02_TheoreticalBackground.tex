\chapter{Theoretical Background}

This chapter summarizes the theoretical concepts required for modelling, linearization,
parameter identification, and control of the Ball-in-Tube system. Only the essential
relations used in the subsequent sections are presented.

\section*{Nonlinear Ball Dynamics}

The motion of the ball is governed by gravity and aerodynamic drag. Assuming that
frictional effects are negligible and the dominant contribution arises from pressure drag,
the vertical dynamics can be expressed as
\[
    m\ddot{z}(t) = F_{\mathrm{w}}(t) - mg,
\]
where \(m\) is the mass of the ball, \(z(t)\) its height, and \(F_{\mathrm{w}}(t)\) the drag force generated
by the upward airflow. Under the assumptions given in the laboratory instructions, the drag
force can be simplified to a quadratic function of the relative velocity between the airflow
and the ball. Introducing the constant coefficients \(\alpha_1, \alpha_2,\alpha_3\), the nonlinear model is
summarized as
\[
    \ddot{z}(t) = \left( \alpha_1 \omega(t) - \alpha_2 \dot{z}(t) \right)^2 - \alpha_3,
\]
where \(\omega(t)\) denotes the fan speed. This equation represents the nonlinear plant model
used for simulation and linearization.

\section*{Fan Dynamics}

The fan exhibits a first-order dynamic behaviour relating the motor voltage \(u(t)\) to the
resulting fan speed \(\omega(t)\). Neglecting inductive effects, the fan can be approximated by
a PT1 system
\[
    \tau_\omega \dot{\omega}(t) + \omega(t)
    = k_\omega \left( u(t) - u_0 \right),
\]
with time constant \(\tau_\omega\), gain \(k_\omega\), and voltage offset \(u_0\). These parameters are
identified experimentally using step-response measurements.

\section*{Linearization}

To enable classical control design, the nonlinear plant is linearized around a chosen
steady-state operating point \((\bar{z},\bar{\dot{z}},\bar{\omega},\bar{u})\). Assuming small perturbations
around this equilibrium, the nonlinear equations are approximated by a linear time-invariant
state-space system. This linear model forms the basis for PID tuning and initial controller
design in MATLAB.

\section*{PID Control}

A PID controller is selected due to its robustness and suitability for single-input single-output
systems with moderate nonlinearities. The controller parameters are tuned using MATLAB’s
\texttt{pidTuner}, taking into account actuator saturation, noise amplification, and the slow
dynamics of the fan. Additional measures such as integrator anti-windup and output
filtering improve the closed-loop performance under realistic conditions.

\section*{Differential Flatness and Trajectory Planning}

The vertical position \(z(t)\) of the ball is a flat output of the nonlinear system, enabling
explicit computation of feed-forward inputs from a reference trajectory. A smooth polynomial
transition is used to generate trajectories that avoid abrupt changes in velocity and
acceleration, thereby improving tracking behaviour when combined with the feedback
controller.
